\documentclass{article}
\usepackage{amsmath, amsthm, amssymb, amsfonts}
\usepackage{ctex}
\usepackage{enumerate}
\usepackage{hyperref}

\theoremstyle{definition}
\newtheorem{theorem}{定理}
\newtheorem{lemma}[theorem]{引理}
\newtheorem{corollary}[theorem]{推论}
\newtheorem{definition}[theorem]{定义}
\newtheorem{example}[theorem]{例子}


\renewcommand{\P}{\mathbb{P}}
\newcommand{\E}{\mathbb{E}}
\newcommand{\Var}{\text{Var}}
\newcommand{\R}{\mathbb{R}}
\newcommand{\1}{\mathbf{1}}
\newcommand{\subg}{\text{Subg}}
\newcommand{\subexp}{\text{Subexp}}

\title{集中不等式总结 }
\author{邓文平}
\date{\today}

\begin{document}

\maketitle

\section{基础不等式}

\begin{theorem}[马尔可夫不等式]
设 \( X \) 为非负随机变量,则对任意 \( t > 0 \):
\[
\P(X \geq t) \leq \frac{\E[X]}{t}.
\]
\end{theorem}

\begin{proof}
通过期望的定义:
\[
\E[X] = \int_0^\infty x d\P_X(x) \geq \int_t^\infty x d\P_X(x) \geq t\int_t^\infty d\P_X(x) = t\P(X \geq t).
\]
两边除以 \( t \) 即得结论。
\end{proof}

\begin{theorem}[切比雪夫不等式]
设 \( X \) 为随机变量,均值为 \( \mu \),方差为 \( \sigma^2 \),则:
\[
\P(|X - \mu| \geq t) \leq \frac{\sigma^2}{t^2}.
\]
\end{theorem}

\begin{proof}
对随机变量 \( Y = (X - \mu)^2 \) 应用马尔可夫不等式:
\[
\P(|X - \mu| \geq t) = \P(Y \geq t^2) \leq \frac{\E[Y]}{t^2} = \frac{\Var(X)}{t^2}.
\]
\end{proof}


\begin{theorem}[切尔诺夫界]
设 \( X = \sum_{i=1}^n X_i \),其中 \( X_i \) 独立。对任意 \( t > 0 \):
\[
\P(X \geq t) \leq \inf_{\lambda > 0} e^{-\lambda t} \E\left[e^{\lambda X}\right].
\]
\end{theorem}

\begin{proof}
对任意 \( \lambda > 0 \),应用马尔可夫不等式于 \( e^{\lambda X} \):
\[
\P(X \geq t) = \P(e^{\lambda X} \geq e^{\lambda t}) \leq e^{-\lambda t}\E[e^{\lambda X}].
\]
取右端关于 \( \lambda \) 的下确界即得结果。
\end{proof}

\section{高斯分布的尾部概率估计}

\begin{theorem}[高斯尾部概率估计]
设 \( X \sim \mathcal{N}(0, \sigma^2) \) 为标准高斯随机变量,则对任意 \( t > 0 \),其尾部概率满足:
\[
\P(X \geq t) \leq \frac{1}{2} e^{-t^2/(2\sigma^2)}.
\]
更精确的双侧估计为:
\[
\P(|X| \geq t) \leq \sqrt{\frac{2}{\pi}} \frac{\sigma}{t} e^{-t^2/(2\sigma^2)} \quad (t > \sigma).
\]
\end{theorem}

\begin{proof}
(单侧估计) 对 \( X \sim \mathcal{N}(0, \sigma^2) \),计算尾部概率:
\[
\P(X \geq t) = \int_t^\infty \frac{1}{\sqrt{2\pi\sigma^2}} e^{-x^2/(2\sigma^2)} dx.
\]
作变量替换 \( y = x/\sigma \),得:
\[
= \int_{t/\sigma}^\infty \frac{1}{\sqrt{2\pi}} e^{-y^2/2} dy.
\]
利用不等式 \( \int_u^\infty e^{-y^2/2} dy \leq \frac{1}{u} e^{-u^2/2} \) (当 \( u > 0 \)),取 \( u = t/\sigma \) 即得:
\[
\P(X \geq t) \leq \frac{\sigma}{t\sqrt{2\pi}} e^{-t^2/(2\sigma^2)} \leq \frac{1}{2} e^{-t^2/(2\sigma^2)}.
\]

(精确双侧估计) 通过分部积分可得递推式:
\[
\int_t^\infty e^{-y^2/2} dy = \frac{e^{-t^2/2}}{t} - \int_t^\infty \frac{e^{-y^2/2}}{y^2} dy \leq \frac{e^{-t^2/2}}{t}.
\]
结合对称性 \( \P(|X| \geq t) = 2\P(X \geq t) \) 即得结果。
\end{proof}



\section{次高斯分布与次指数分布}

\begin{definition}[次高斯分布]
随机变量 \( X \) 称为参数为 \( \sigma \) 的\textbf{次高斯分布}(记作 \( X \sim \subg(\sigma^2) \)),如果满足:
\[
\E\left[e^{\lambda (X - \E X)}\right] \leq e^{\frac{\lambda^2 \sigma^2}{2}}, \quad \forall \lambda \in \R.
\]
\end{definition}

\begin{proof}[尾概率证明]
对任意 \( t > 0 \),取 \( \lambda = t/\sigma^2 \),应用切尔诺夫界:
\[
\P(X - \E X \geq t) \leq e^{-\lambda t} e^{\lambda^2 \sigma^2/2} = e^{-t^2/(2\sigma^2)}.
\]
结合 \( -X \) 的情况即得双侧界。
\end{proof}

\section{霍夫丁不等式}

\begin{theorem}[对称伯努利变量的霍夫丁不等式]
设 \( X_1, \dots, X_n \) 为独立对称伯努利随机变量(即 \( \P(X_i = 1) = \P(X_i = -1) = \frac{1}{2} \)),令 \( S_n = \sum_{i=1}^n X_i \)。则对任意 \( t > 0 \):
\[
\P(S_n \geq t) \leq \exp\left(-\frac{t^2}{2n}\right).
\]
双侧形式为:
\[
\P(|S_n| \geq t) \leq 2\exp\left(-\frac{t^2}{2n}\right).
\]
\end{theorem}

\begin{proof}
\textbf{方法一}(从一般霍夫丁不等式导出):
由于 \( X_i \in [-1,1] \) 且 \( \E X_i = 0 \),应用标准霍夫丁不等式:
\[
\sum_{i=1}^n (b_i - a_i)^2 = \sum_{i=1}^n (1 - (-1))^2 = 4n.
\]
代入得:
\[
\P(S_n \geq t) \leq \exp\left(-\frac{2t^2}{4n}\right) = \exp\left(-\frac{t^2}{2n}\right).
\]

\textbf{方法二}(直接矩生成函数计算):
计算单个变量的矩生成函数:
\[
\E[e^{\lambda X_i}] = \cosh(\lambda) \leq e^{\lambda^2/2}.
\]
利用独立性:
\[
\E[e^{\lambda S_n}] = \prod_{i=1}^n \E[e^{\lambda X_i}] \leq e^{n\lambda^2/2}.
\]
应用切尔诺夫界:
\[
\P(S_n \geq t) \leq \inf_{\lambda > 0} e^{-\lambda t} e^{n\lambda^2/2} = e^{-t^2/(2n)}.
\]
其中最优 \( \lambda = t/n \)。
\end{proof}

\begin{corollary}[次高斯性验证]
对称伯努利变量 \( X_i \) 是 \( \subg(1) \) 的,其和 \( S_n \sim \subg(n) \)。
\end{corollary}

\begin{proof}
由矩生成函数估计:
\[
\E[e^{\lambda X_i}] = \frac{e^\lambda + e^{-\lambda}}{2} = \cosh(\lambda) \leq e^{\lambda^2/2}.
\]
和变量满足 \( \E[e^{\lambda S_n}] \leq e^{n\lambda^2/2} \),符合次高斯性定义。
\end{proof}

\begin{theorem}[霍夫丁不等式]
设 \( X_1, \dots, X_n \) 为独立随机变量,满足 \( X_i \in [a_i, b_i] \) 几乎必然成立。令 \( S_n = \sum_{i=1}^n (X_i - \E X_i) \),则对任意 \( t > 0 \):
\[
\P(S_n \geq t) \leq \exp\left(-\frac{2t^2}{\sum_{i=1}^n (b_i - a_i)^2}\right).
\]
\end{theorem}

\begin{proof}
关键步骤:
1. 对每个 \( X_i - \E X_i \) 建立次高斯性,其参数 \( \sigma_i = (b_i - a_i)/2 \)
2. 利用独立性得 \( S_n \) 的矩生成函数:
\[
\E[e^{\lambda S_n}] = \prod_{i=1}^n \E[e^{\lambda (X_i - \E X_i)}] \leq \prod_{i=1}^n e^{\lambda^2 (b_i - a_i)^2/8}
\]
3. 应用切尔诺夫界并优化 \( \lambda \)
\end{proof}

\section{基于鞅的不等式}

\begin{theorem}[Azuma-Hoeffding不等式]
设 \( \{X_i\}_{i=0}^n \) 为鞅序列,满足 \( |X_i - X_{i-1}| \leq c_i \) 几乎必然成立。则对任意 \( t > 0 \):
\[
\P(|X_n - X_0| \geq t) \leq 2 \exp\left(-\frac{t^2}{2 \sum_{i=1}^n c_i^2}\right).
\]
\end{theorem}

\begin{proof}
构造鞅差序列 \( D_i = X_i - X_{i-1} \),利用霍夫丁引理证明每个 \( D_i \) 的次高斯性,再通过矩生成函数的乘积性质完成证明。
\end{proof}

\section{应用说明}
\begin{itemize}
\item 次高斯性是推导高斯向量Lipschitz函数尖锐界限的关键
\item McDiarmid不等式在机器学习中广泛用于算法稳定性分析
\item Azuma不等式适用于依赖过程(例如具有依赖性的随机游走)
\end{itemize}

\end{document}