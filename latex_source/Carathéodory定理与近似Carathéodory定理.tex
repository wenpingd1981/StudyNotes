\documentclass{article}
\usepackage{amsmath}
\usepackage{amssymb}
\usepackage{amsthm}
\usepackage{graphicx}
\usepackage{hyperref}
\usepackage{ctex} % 支持中文
\usepackage[utf8]{inputenc}
\usepackage[T1]{fontenc}

% 定义 theorem 和 definition 环境
\newtheorem{theorem}{定理}
\newtheorem{definition}{定义}

\title{Carathéodory定理与近似Carathéodory定理}
\author{邓文平}
\date{\today}

\begin{document}

\maketitle

\section{Carathéodory定理}

\textbf{定理陈述:} 在\(\mathbb{R}^n\)中,若点\( x \)属于集合\( S \)的凸包,则\( x \)可以表示为\( S \)中至多\( n+1 \)个点的凸组合。

\textbf{证明过程:}

\subsection{凸组合表示:}
设\( x \in \operatorname{conv}(S) \),则存在有限个点\( x_1, x_2, \ldots, x_k \in S \)和系数\( \lambda_1, \lambda_2, \ldots, \lambda_k > 0 \),使得

\[
x = \sum_{i=1}^{k} \lambda_i x_i, \quad \sum_{i=1}^{k} \lambda_i = 1.
\]

若\( k \leq n+1 \),结论已成立。假设\( k > n+1 \),需将\( k \)缩减至\( n+1 \)。

\subsection{线性相关性:}
考虑向量\(\{ (x_i, 1) \}_{i=1}^{k} \subset \mathbb{R}^{n+1}\)。由于\( k > n+1 \),这组向量线性相关,故存在不全为零的实数\( \alpha_1, \alpha_2, \ldots, \alpha_k \),使得

\[
\sum_{i=1}^{k} \alpha_i x_i = 0 \quad \text{且} \quad \sum_{i=1}^{k} \alpha_i = 0.
\]

\subsection{系数调整:}
定义新系数\( \lambda'_i = \lambda_i + t \alpha_i \),其中\( t \in \mathbb{R} \)。此时,

\[
\sum_{i=1}^{k} \lambda'_i x_i = \sum_{i=1}^{k} (\lambda_i + t \alpha_i) x_i = x + t \cdot 0 = x,
\]

且

\[
\sum_{i=1}^{k} \lambda'_i = \sum_{i=1}^{k} (\lambda_i + t \alpha_i) = 1 + t \cdot 0 = 1.
\]

需选择\( t \)使得所有\( \lambda'_i \geq 0 \),且至少有一个\( \lambda'_i = 0 \)。

\subsection{确定参数\( t \):}
令\( t \)满足:

\[
t = \min \left\{ -\frac{\lambda_i}{\alpha_i} \, \bigg| \, \alpha_i < 0 \right\} \quad \text{或} \quad t = \max \left\{ \frac{-\lambda_i}{\alpha_i} \, \bigg| \, \alpha_i > 0 \right\}.
\]

由于\( \sum \alpha_i = 0 \)且\( \alpha_i \)不全为零,必存在正负\( \alpha_i \)。选择上述\( t \)时,至少有一个\( \lambda'_i = 0 \),其余\( \lambda'_i \geq 0 \),从而减少一个非零系数。

\subsection{递归缩减:}
重复上述过程,每次将\( k \)减少1,直到\( k = n+1 \)。此时无法进一步缩减,因为\( n+1 \)个点在\(\mathbb{R}^{n+1}\)齐次坐标下可能线性无关。

\section{近似Carathéodory定理}

\textbf{定理陈述:} 设集合\( K \subset \mathbb{R}^n \),且所有点满足\( \| v \|_2 \leq R \)。对于任意\( x \in \operatorname{conv}(K) \)和\( \epsilon > 0 \),存在\( x' \in \operatorname{conv}(K) \),使得:

\( x' \)是\( K \)中至多\( d = \left\lceil \frac{R^2}{\epsilon^2} \right\rceil \)个点的凸组合;

\( \| x - x' \|_2 \leq \epsilon \)。

\textbf{证明步骤:}

\subsection{凸组合表示:}
由Carathéodory定理,\( x \)可表示为\( K \)中有限个点的凸组合。设存在点\( v_1, v_2, \ldots, v_k \in K \)及系数\( \lambda_i \geq 0 \),满足:

\[
x = \sum_{i=1}^{k} \lambda_i v_i, \quad \sum_{i=1}^{k} \lambda_i = 1.
\]

\subsection{随机采样构造:}
定义随机向量\( V \),以概率\( \lambda_i \)取值为\( v_i \)。独立采样\( d \)次得到\( V_1, V_2, \ldots, V_d \),构造经验均值:

\[
x' = \frac{1}{d} \sum_{j=1}^{d} V_j.
\]

显然,\( x' \)是\( K \)中至多\( d \)个点的凸组合。

\subsection{期望与方差分析:}

\textit{期望:} 由线性性,

\[
\mathbb{E}[x'] = \mathbb{E}\left[ \frac{1}{d} \sum_{j=1}^{d} V_j \right] = \frac{1}{d} \sum_{j=1}^{d} \mathbb{E}[V_j] = \mathbb{E}[V] = x.
\]

\textit{方差:} 计算二阶矩:

\[
\mathbb{E}[\| x' - x \|_2^2] = \mathbb{E}\left[ \left\| \frac{1}{d} \sum_{j=1}^{d} (V_j - x) \right\|_2^2 \right].
\]

由于\( V_j \)独立,

\[
\operatorname{Var}(x') = \frac{1}{d^2} \sum_{j=1}^{d} \operatorname{Var}(V_j) = \frac{1}{d} \operatorname{Var}(V).
\]

因此,

\[
\mathbb{E}[\| x' - x \|_2^2] = \frac{1}{d} \mathbb{E}[\| V - x \|_2^2].
\]


因为 
\[
\mathbb{E}[\| V_1 - V_2 \|_2^2] = \mathbb{E}[\|( V_1- \mathbb{E}[V_1])- (V_2- \mathbb{E}[V_1]) \|_2^2]  = 2\mathbb{E}[\| V - x \|_2^2]
\]

所以 

\[
\mathbb{E}[\| V - x \|_2^2] = \frac{1}{2}  \mathbb{E}[\| V_1 - V_2 \|_2^2]
\]

又由于 \( \| V \|_2 \leq R \),

\[
 \mathbb{E}[\| V - x \|_2^2] \leq   \frac{1}{2} (2R)^2 =2R^2
\]


所以,

\[
\mathbb{E}[\| x' - x \|_2^2] \leq \frac{2R^2}{d}.
\]

\subsection{存在性论证:}
由期望不等式,存在至少一个具体的\( x' \)满足:

\[
\| x' - x \|_2^2 \leq \frac{2R^2}{d}.
\]

取\( d = \left\lceil \frac{2R^2}{\epsilon^2} \right\rceil \),则:

\[
\| x' - x \|_2 \leq \sqrt{ \frac{2R^2}{d} } \leq \epsilon.
\]

\end{document}
